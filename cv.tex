% !TEX root = ./cv_korean_style.tex
\documentclass[a4paper,10pt]{article}
\usepackage[left=0.75in,top=0.6in,right=0.75in,bottom=0.6in]{geometry}
\usepackage{hyperref}
\usepackage{enumitem}
\usepackage{titlesec}
\usepackage{tabularx}
\usepackage{graphicx}
\usepackage{kotex} % 한글 출력을 위해 필수

% --- 커스텀 스타일 정의 (기존 양식 유지) ---
\pagestyle{empty} % 페이지 번호 제거
\raggedright % 왼쪽 정렬

% 섹션 스타일
\titleformat{\section}{\large\bfseries\uppercase}{}{0em}{}[\titlerule]
\titlespacing{\section}{0pt}{12pt}{6pt}

% 항목 스타일 (제목 + 날짜)
\newcommand{\entry}[4]{
  \textbf{#1} \hfill #2 \\
  \textit{#3} \hfill \textit{#4}
}

% 프로젝트 스타일 (제목 + 날짜 / 역할)
\newcommand{\project}[3]{
  \textbf{#1} \hfill #2 \\
  \textit{#3}
}

% 리스트 설정
\setlist[itemize]{leftmargin=1.5em, noitemsep, topsep=2pt}

% 링크 색상 설정
\hypersetup{
    colorlinks=true,
    linkcolor=black,
    filecolor=magenta,      
    urlcolor=blue,
}

\begin{document}

% --- 헤더 (이름 및 연락처) ---
\begin{center}
    % 텍스트 왼쪽, 사진 오른쪽 배치: public/assets/photo.jpg 위치에 이미지 파일을 저장하세요.
    \begin{minipage}[c]{0.65\textwidth}
        {\Huge \textbf{손 호 영}} \\ [4pt]
        2002.09.05 (Age 23) \\ [2pt]
        010-8586-4134 \\ [2pt]
        \href{mailto:tls1234568@naver.com}{tls1234568@naver.com} \\ [2pt]
        \href{https://github.com/pyrimidine02}{github.com/pyrimidine02}
    \end{minipage}\hfill
    \begin{minipage}[c]{3.2cm}
        \includegraphics[width=3.0cm,height=3.0cm,keepaspectratio]{public/assets/photo.jpg}
    \end{minipage}
\end{center}

\vspace{10pt}

% --- EDUCATION ---
\section{Education}

\entry{인하대학교 (Inha University)}{인천, 대한민국}{컴퓨터공학과 학사 (재학)}{2023.02 -- 현재}

\entry{Google for Developers Machine Learning Bootcamp 2024}{온라인}{수료}{2024}

% --- EXPERIENCE ---
\section{Experience}

\project{Portable ECG Monitor on Edge Device using TensorFlow Lite}{ISIPS 2023}{논문 기여 및 연구 참여}
\begin{itemize}
    \item 부정맥(ECG) 데이터 전처리 수행.
    \item 경량 모델(TensorFlow Lite) 전처리, 학습 및 검증 과정에 기여.
\end{itemize}

% --- PROJECTS ---
\section{Projects}

\project{RailNetwork Control}{2023.07 -- 현재}{Lead Engineer}
\begin{itemize}
    \item 8,852명 활성 사용자가 사용하는 서울 지하철 실시간 추적 서비스 TrainAccAPI 백엔드 설계·구축.
    \item 20개 노선, 2,000여 대 열차를 15초 주기로 실시간 추적하며, 25M+ 이벤트 처리.
    \item 코레일 API 복원 및 폴백 시스템 구축, 평균 응답시간 150ms, 크로스 플랫폼 서비스 제공.
    \item Tech: Node.js, Express.js, gRPC, MongoDB, Redis, Winston. \href{https://www.railnetwork.kr/}{서비스}
\end{itemize}

\vspace{6pt}

\project{Girls Band Tabi LBS Platform}{2025.08 -- 현재}{Founding Engineer}
\begin{itemize}
    \item Kotlin + Spring Boot 3 + PostGIS 기반 위치기반 서비스로 밴드 성지순례 플랫폼 설계·구축.
    \item 180+ REST API 엔드포인트, PostGIS 공간 쿼리 최적화, JWT/RBAC 다층 보안 구현.
    \item 프로젝트·유닛·지역 계층 구조와 DDD 모듈 아키텍처로 멀티테넌트 시스템 설계.
    \item Tech: Kotlin, Spring Boot 3, PostgreSQL 15, PostGIS 3.3, Redis. \href{https://github.com/pyrimidine02/oshitabi}{깃허브}
\end{itemize}

% --- HONORS AND AWARDS ---
\section{Honors and Awards}
\begin{tabularx}{\textwidth}{@{}l X r@{}}
\textbf{최종 11위} & Dacon 2024 인하 인공지능 챌린지 & 2024.08 \\
\textbf{Top 1.3\%} & REPLY Code Challenge 2024 (1,844팀 중 24등) & 2024.03 \\
\textbf{TOP 6} & 빅데이터·AI 경진대회 (AWS, KT AICE 공동 주관) & 2023.08 \\
\textbf{Top 3.7\%} & Kaggle Binary Classification of Insurance Cross Selling (83등/2,236팀) & 2024.11 \\
\textbf{교내 3등} & ICPC Seoul Regional 교내 예선 & 2023.10 \\
\end{tabularx}

% --- CLUB ACTIVITIES ---
\section{Club Activities}

\project{IGRUS, INGAME}{2023.09 -- 2024.07}{문제해결 강의 및 멘토링}
\begin{itemize}
    \item 소속 동아리(IGRUS, INGAME)에서 알고리즘 소모임장을 맡아 활동.
    \item 1년간 40여 명의 학부생을 대상으로 C++, Python 언어 및 알고리즘 기초 강의 진행.
    \item 2023 IGRUS Newbie Programming Contest 대회를 개최 및 문제 검수.
\end{itemize}
\vspace{4pt}

\project{CTP (알고리즘 동아리)}{2024.01 -- 현재}{운영진 (총무)}
\begin{itemize}
    \item 2024년 1학기부터 운영진으로 참여.
    \item 코딩테스트 대비반 알고리즘 강의 진행 및 총무 담당.
\end{itemize}
\vspace{4pt}

\project{밴드 동아리}{}{기타(Guitar) 세션 담당}

% --- TECHNICAL SKILLS ---
\section{Technical Skills}
\begin{itemize}
    \item \textbf{Core Languages:} Python, Kotlin, JavaScript (Node.js), C++, Flutter
    \item \textbf{Backend Frameworks:} Spring Boot 3, Express.js, gRPC
    \item \textbf{Database \& Storage:} PostgreSQL/PostGIS, MongoDB, Redis, Flyway
    \item \textbf{Cloud \& DevOps:} Docker, Cloudflare R2, AWS, HikariCP
    \item \textbf{Security \& Auth:} JWT, Spring Security, HMAC-SHA256
    \item \textbf{Machine Learning:} TensorFlow Lite, PyTorch, scikit-learn
    \item \textbf{Problem Solving:}
    \begin{itemize}[label=$\circ$]
        \item Solved.ac (Baekjoon): Platinum IV (Rating: 1774)
        \item Codeforces: Pupil (Rating: 1214)
    \end{itemize}
\end{itemize}

\end{document}