% !TEX root = ./cv_korean_style.tex
\documentclass[a4paper,10pt]{article}
\usepackage[left=0.75in,top=0.6in,right=0.75in,bottom=0.6in]{geometry}
\usepackage{hyperref}
\usepackage{enumitem}
\usepackage{titlesec}
\usepackage{tabularx}
\usepackage{graphicx}
\usepackage{kotex} % 한글 출력을 위해 필수

% --- 커스텀 스타일 정의 (기존 양식 유지) ---
\pagestyle{empty} % 페이지 번호 제거
\raggedright % 왼쪽 정렬

% 섹션 스타일
\titleformat{\section}{\large\bfseries\uppercase}{}{0em}{}[\titlerule]
\titlespacing{\section}{0pt}{12pt}{6pt}

% 항목 스타일 (제목 + 날짜)
\newcommand{\entry}[4]{
  \textbf{#1} \hfill #2 \\
  \textit{#3} \hfill \textit{#4}
}

% 프로젝트 스타일 (제목 + 날짜 / 역할)
\newcommand{\project}[3]{
  \textbf{#1} \hfill #2 \\
  \textit{#3}
}

% 리스트 설정
\setlist[itemize]{leftmargin=1.5em, noitemsep, topsep=2pt}

% 링크 색상 설정
\hypersetup{
    colorlinks=true,
    linkcolor=black,
    filecolor=magenta,
    urlcolor=blue,
}

\begin{document}

% --- 헤더 (이름 및 연락처) ---
\begin{center}
    % 텍스트 왼쪽, 사진 오른쪽 배치: public/assets/photo.jpg 위치에 이미지 파일을 저장하세요.
    \begin{minipage}[c]{0.65\textwidth}
        {\Huge \textbf{손 호 영}} \\ [4pt]
        2002.09.05 (Age 23) \\ [2pt]
        010-8586-4134 \\ [2pt]
        \href{mailto:tls1234568@naver.com}{tls1234568@naver.com} \\ [2pt]
        \href{https://github.com/pyrimidine02}{github.com/pyrimidine02}
    \end{minipage}\hfill
    \begin{minipage}[c]{3.2cm}
        \includegraphics[width=3.0cm,height=3.0cm,keepaspectratio]{public/assets/photo.jpg}
    \end{minipage}
\end{center}

\vspace{6pt}

% --- SUMMARY ---
\begin{center}
\textit{6,375명 사용자·1,400+ DAU 프로덕션 서비스를 설계·운영하는 백엔드 엔지니어.} \\[2pt]
\textit{리버스 엔지니어링 기반 API 복원, 이벤트 드리븐 모듈 아키텍처, 이중 캐시 설계를 통해 실 트래픽 문제를 해결합니다.}
\end{center}

\vspace{4pt}

% --- PROJECTS ---
\section{Projects}

\project{RailNetwork}{2023.07 -- 현재}{Lead Engineer}
\begin{itemize}
    \item 공식 API 미제공 환경에서 Proxyman 리버스엔지니어링으로 SMSS API를 복원하고, SMSS 장애 시 열린데이터 API로 자동 전환하는 이중 폴백 시스템 구축.
    \item Promise.all 병렬 처리로 전체 노선 동시 업데이트, Redis 캐싱으로 응답 성능 최적화.
    \item 20개 노선 2,000여 대 열차를 15초 주기로 실시간 추적하는 백엔드 설계·구축, \textbf{6,375명 사용자·12M+ 이벤트} 처리. (2025.11--2026.02 기준)
    \item Firebase Analytics 지역 분석(서울 68\%, 경기 22\%)을 통한 데이터 기반 의사결정, 일 평균 \textbf{1,400+ DAU}.
    \item 크라우드소싱 편성 투표 시스템 설계, Flutter 기반 Android/iOS 크로스 플랫폼 제공.
    \item Tech: Node.js, Express.js, gRPC, MongoDB, Redis, JWT, HMAC-SHA256, Flutter. \href{https://www.railnetwork.kr/}{서비스}
\end{itemize}

\vspace{6pt}

\project{Girls Band Tabi \textnormal{\small(서버 운영 중, 앱 내부 테스트)}}{2025.08 -- 현재}{Founding Engineer}
\begin{itemize}
    \item Kotlin + Spring Boot 3 + Spring Modulith 이벤트 드리븐 아키텍처로 5개 도메인(identity·place·event·social·analytics)을 느슨한 결합으로 분리한 위치 기반 플랫폼 설계·구축.
    \item OAuth2(Google/Apple) + JWT + 이메일 인증 다중 인증 체계, Resilience4j 서킷 브레이커 + Anti-tamper 위치 검증으로 보안 다층 방어.
    \item Caffeine L1 / Redis L2 이중 캐시 + cache-aside 패턴으로 읽기 최적화, 도메인 단위 캐시 키로 정밀 무효화 설계.
    \item Cloudflare R2 비용 보호(일일 대역폭/월간 요청 제한), Flyway 마이그레이션, HikariCP 커넥션 풀 최적화로 인프라 안정성 확보.
    \item Detekt/Ktlint 정적 분석 + Kover 커버리지(서비스 계층 80\% 목표) CI 파이프라인 통합, 코드 품질 게이트 자동화.
    \item Tech: Kotlin, Spring Boot 3, Spring Modulith, PostgreSQL, Hibernate Spatial, Redis, Resilience4j, Prometheus. \href{https://github.com/pyrimidine02/girlsbandtabi}{깃허브}
\end{itemize}

% --- TECHNICAL SKILLS ---
\section{Technical Skills}
\begin{itemize}
    \item \textbf{Languages:} Kotlin, Python, JavaScript (Node.js), C++, Dart (Flutter)
    \item \textbf{Backend:} Spring Boot 3, Spring Modulith, Express.js, gRPC, Resilience4j
    \item \textbf{Database \& Cache:} PostgreSQL/Hibernate Spatial, MongoDB, Redis, Flyway, HikariCP
    \item \textbf{Infra \& DevOps:} Docker, Cloudflare R2, AWS, Prometheus, Testcontainers, CI/CD (Detekt/Ktlint/Kover)
    \item \textbf{Problem Solving:}
    \begin{itemize}[label=$\circ$]
        \item Solved.ac (Baekjoon): Platinum IV (Rating: 1774)
        \item Codeforces: Pupil (Rating: 1214)
    \end{itemize}
\end{itemize}

% --- HONORS AND AWARDS ---
\section{Honors and Awards}
\begin{tabularx}{\textwidth}{@{}l X r@{}}
\textbf{최종 11위} & Dacon 2024 인하 인공지능 챌린지 & 2024.08 \\
\textbf{Top 1.3\%} & REPLY Code Challenge 2024 (1,844팀 중 24등) & 2024.03 \\
\textbf{TOP 6} & 빅데이터·AI 경진대회 (AWS, KT AICE 공동 주관) & 2023.08 \\
\textbf{Top 3.7\%} & Kaggle Binary Classification of Insurance Cross Selling (83등/2,236팀) & 2024.11 \\
\textbf{교내 3등} & ICPC Seoul Regional 교내 예선 & 2023.10 \\
\end{tabularx}

% --- EDUCATION ---
\section{Education}

\entry{인하대학교 (Inha University)}{인천, 대한민국}{컴퓨터공학과 학사 (재학)}{2023.02 -- 현재}

\entry{Google for Developers Machine Learning Bootcamp 2024}{온라인}{수료}{2024}

% --- EXPERIENCE ---
\section{Experience}

\project{Portable ECG Monitor on Edge Device using TensorFlow Lite}{ISIPS 2023}{데이터 전처리, 모델링, 모델 경량화 담당}
\begin{itemize}
    \item MIT-BIH 부정맥 데이터베이스에서 QRS 파형 정점 식별, 96$\times$96 이미지 리사이징, R-wave 중심 정렬 및 그레이스케일 변환 등 ECG 데이터 전처리 파이프라인 구축.
    \item CNN 모델(Conv2 + Pool2 + FC) 설계 및 학습, TensorFlow Lite 8-bit 양자화로 모델 크기를 2.4\,MB에서 503\,KB로 경량화하여 Arduino Nano 33 BLE에 배포.
    \item 개인 ECG 데이터 재학습으로 TF Lite 모델 정확도를 53.75\%에서 83.55\%로 개선.
\end{itemize}

% --- CLUB ACTIVITIES ---
\section{Club Activities}

\project{IGRUS, INGAME}{2023.09 -- 2024.07}{문제해결 강의 및 멘토링}
\begin{itemize}
    \item 소속 동아리(IGRUS, INGAME)에서 알고리즘 소모임장을 맡아 활동.
    \item 1년간 40여 명의 학부생을 대상으로 C++, Python 언어 및 알고리즘 기초 강의 진행.
    \item 2023 IGRUS Newbie Programming Contest 대회를 개최 및 문제 검수.
\end{itemize}
\vspace{4pt}

\project{CTP (알고리즘 동아리)}{2024.01 -- 현재}{운영진 (총무)}
\begin{itemize}
    \item 2024년 1학기부터 운영진으로 참여.
    \item 코딩테스트 대비반 알고리즘 강의 진행 및 총무 담당.
\end{itemize}
\vspace{4pt}

\project{AiBO!!!!! (BanG Dream! MyGO!!!!! 커버 밴드)}{}{기타(Guitar) 세션 담당}

\end{document}
